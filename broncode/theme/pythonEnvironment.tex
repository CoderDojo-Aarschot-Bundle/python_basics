\usepackage{xcolor}
\usepackage{transparent}
\usepackage{pgf}
\usepackage{listings}
\usepackage{tikz}


\definecolor{darkBlue}{rgb}{0,0,0.5}
\definecolor{darkGreen}{rgb}{0, 0.5, 0}
\definecolor{darkRed}{rgb}{0.5, 0, 0}
\definecolor{wineRed}{rgb}{0.7, 0.1, 0.2}
\definecolor{darkCyan}{rgb}{0, 0.5, 0.5}

\renewcommand{\ttdefault}{pcr}                                                  % use Courier font instead of Computer Modern
																				% because it has bold face

% python style for highlighting
\lstdefinestyle{pythonstyle}{
	language=Python,
	basicstyle=\small\ttfamily\color{blue},                                     % applicable to numbers since all other items will get its specific styles defined
	identifierstyle =\ttfamily\color{black},
	literate=
		% other identifiers
		{:}{{\textcolor{black}{:}}}1
		{;}{{\textcolor{black}{;}}}1
		{(}{{\textcolor{black}{(}}}1
		{)}{{\textcolor{black}{)}}}1
		{<}{{\textcolor{black}{<}}}1
		{>}{{\textcolor{black}{>}}}1
		{>>>}{{\textcolor{gray}{>>>}}}3
		{=}{{\textcolor{black}{=}}}1
		{+}{{\textcolor{black}{+}}}1
		{-}{{\textcolor{black}{-}}}1
		{*}{{\textcolor{black}{*}}}1
		{/}{{\textcolor{black}{/}}}1
		{\%}{{\textcolor{black}{\%}}}1
		{&}{{\textcolor{black}{&}}}1
		{|}{{\textcolor{black}{|}}}1,
	commentstyle=\color{gray},
	keywordstyle=\bfseries\color{darkBlue},
	morekeywords={self,None},
	% boolean values
	morekeywords=[3]{True,False},	 
	keywordstyle=[3]\color{darkGreen},
	emph=[3]{bool},	 
	emphstyle=[3]\bfseries\color{darkGreen},
	% integer values
	emph=[4]{int},	 
	emphstyle=[4]\bfseries\color{blue},
	% float values
	emph=[5]{float},	 
	emphstyle=[5]\bfseries\color{orange},
	% there is no direct way to differentiate between float and int in lstlisting
	% the only thing that is possible is to apply a different style to the dot, which only appears in floats
	% literate=
	% 	{.}{{\textcolor{orange}{.}}}1,
	% another option is to use markers to differentiate floats from ints
	% it's not perfect because it requires an extra action from the user
	% but it will work towards the generated pdf
	moredelim=[is][\bfseries\color{orange}]{@float}{@},
	% string values
	% morekeywords=[6]{str},	 
	% keywordstyle=[6]\color{wineRed},
	emph=[6]{str},
	emphstyle=[6]\bfseries\color{wineRed},
	stringstyle=\color{wineRed},
	% errors
	emph=[7]{NameError,IndentationError},	 
	emphstyle=[7]\bfseries\color{darkRed},
	% optional syntax highlighting
	moredelim=[is][\itshape\transparent{0.35}]{@optional}{@},
	% other styling options
	frame=ltrb,
	framexleftmargin=0.7em,
	xleftmargin=2em,
	framexrightmargin=0.7em,
	xrightmargin=2em,
	backgroundcolor=\color[rgb]{.96, 1, 1},
	% numbers=left,
	% numberstyle=\tiny\color{gray},
	keepspaces=true,
	% showtabs=true,
	tabsize=4,
	showstringspaces=false,
	columns=fullflexible,                                                       % copy-pastable code, preserving exact amount of spaces
	                                                                            % but ... https://stackoverflow.com/a/3529488
	escapechar=$,
}

% python environment
\newcommand{\highlight}{\makebox[0pt]{
	% looks like text will be centered in box if overflow exists
	% so solution
	% -> create two backgrounds and have the box located at start of line
	%    the second bg will be placed just behind the line
	\color{white}\pgfsetfillopacity{0  }\rule[-1ex]{\linewidth}{3ex}
	\color{green}\pgfsetfillopacity{0.3}\rule[-1ex]{\linewidth}{3ex}
	\pgfsetfillopacity{1}                                                   % restore value
}}
\newenvironment{Separator}[1][0pt]{\parskip=#1\par}{\par\noindent}
\lstnewenvironment{pyEnv}{
	\lstset{style=pythonstyle}
	\begin{Separator}[0.7em]
	\end{Separator}
	\minipage{\textwidth}
}{
	\endminipage
	\begin{Separator}[0.2em]
	\end{Separator}
}

% python for inline
\newcommand{\codeInline}[2][2pt]{\tikz[baseline=(X.base)]\node [draw=darkgray, rectangle, fill=white, inner sep=#1, rounded corners=1pt] (X) {#2};}
% \newcommand{\pySnip}[2][2pt]{\codeInline[#1]{\pythonstyle\lstinline!#2!}}
\newcommand{\pySnip}[2][2pt]{\codeInline[#1]{\lstinline[style=pythonstyle]!#2!}}

% % python for external files
% \newcommand\pythonexternal[2][]{{
% \pythonstyle
% \lstinputlisting[#1]{#2}}}

\newcommand{\datatype}[2][darkCyan]{
	\codeInline{\textcolor{#1}{\texttt{#2}}}
}
% \newcommand{\booleantype}{\datatype[darkGreen]{bool}}
% \newcommand{\integertype}{\datatype[blue]{int}}
% \newcommand{\floattype}{\datatype[orange]{float}}
% \newcommand{\stringtype}{\datatype[wineRed]{str}}
\newcommand{\booleantype}{\pySnip{bool}}
\newcommand{\integertype}{\pySnip{int}}
\newcommand{\floattype}{\pySnip{float}}
\newcommand{\stringtype}{\pySnip{str}}

% \newcommand{\value}[2][darkCyan]{
% 	\textcolor{#1}{
% 		\texttt{#2}
% 	}
% }
\newcommand{\true}{\pySnip{True}}
\newcommand{\false}{\pySnip{False}}
\newcommand{\errorType}[1]{\texttt{\datatype[darkRed]{#1}}}