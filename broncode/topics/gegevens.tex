\section{Gegevens}

Doorheen een computerporgrammaatje gaan we bijna altijd met gegevens zitten goochelen.
Maar wat zijn gegevens nu juist?
En kunnen we zomaar van het ene type naar het andere gaan?

\subsection{Verschillende soorten}

Laten we beginnen met een eerste kennismaking met de meest voorkomend soorten gegevens.

\subsubsection{Text: \stringtype}

De eerste soort gegevens die we tegenkomen zijn teksten.
Een tekst is een reeks van karakters.
Een karakter is een letter, cijfer, leesteken, spatie, ... .
We beginnen en eindigen tekst altijd met aanhalingstekens (\verb|"|).
\begin{pyEnv}
"Dit is een tekst"
"47"                        # ook dat is tekst
\end{pyEnv}

\subsubsection{Getallen: \integertype en \floattype}

Een volgende soort gegevens zijn getallen.
We maken een onderscheid tussen getallen zonder en met komma:
\begin{itemize}
\item Getallen zonder komma noemen we integers (het Engelse woord voor gehele gatellen): \integertype.
\item Getallen met komma noemen we floats (het Engelse woord voor 'vlottende' komma): \floattype.
\end{itemize}
\begin{pyEnv}
47                          # integer
@float47.0@                        # opgelet, een float!
@float3.14@                        # float
\end{pyEnv}

\begin{letsTryOut}
	In python kan je van iedere waarde opvragen tot welke soort gegevens het hoort.
	Hiervoor gebruik je de functie \pySnip{type()}.
	\newline
	Weet jij het verschil tussen \pySnip{"47"}, \pySnip{47} en \pySnip{FL47.0T}?
	\newline
	Probeer het eens in de interactive shell!
\begin{pyEnv}
>>> type("47")
>>> type(47)
>>> type(@float47@)
\end{pyEnv}
\end{letsTryOut}

\subsubsection{Logische waarde: \booleantype}

Een andere veelvoorkomende soort gegevens zijn logische waarden.
Er bestaan exact twee logische waarden:
\begin{itemize}
	\item \true: de uitkomst is juist.
	\item \false: de uitkomst is fout.
\end{itemize}
\begin{pyEnv}
True
False
# dit zijn de enige twee logische waardes die bestaan
\end{pyEnv}

\subsection{Transformaties}

Het is heel belangrijk dat je je realiseert dat python de verschillende soorten gegevens anders behandelt.
Ook al lijken ze voor jou op elkaar, zo python ziet een groot verschil tussen \pySnip{"47"} en \pySnip{47}.
Gelukkig kunnen we meestal heel eenvoudig een waarde van de ene soort omzetten naar een waarde van een andere soort.
Dit doen we met functies die dezelfde naam hebben als de verschillende soorten gegevens.  % I know, those aren't functions
                                                                                          % but it's a children's story ...
Namelijk:
\begin{itemize}
	\item \pySnip{str(_)}: een waarde omzetten naar tekst
	\item \pySnip{int(_)}: een waarde omzetten naar getal (zonder komma)
	\item \pySnip{float(_)}: een waarde omzetten naar kommagetal
	\item \pySnip{bool(_)}: een waarde omzetten naar logische waarde
\end{itemize}

\begin{letsTryOut}
	We hebben deze twee waardes verkregen van de gebruiker via de functie \pySnip{input()}:
	\pySnip{"12"} en \pySnip{"36"}.
	\newline
	% Merk op dat wat je krijgt van \pySnip{input()} altijd tekst is.
	We willen de gebruiker de som (\pySnip{+}) van deze twee getallen tonen.
	We kunnen dat rechtsreeks met de gekregen tekst doen.
	Of we kunnen de waardes eerst omvormen naar getallen.
	\newline
	Is er een verschil, denk je?
	\newline
	Had je het juist? Kan je uitleggen waarom?
\begin{pyEnv}
>>> "12" + "36"
>>> int("12") + int("36")
\end{pyEnv}
\end{letsTryOut}

\begin{letsTryOut}
	Laten we nog eens iets anders proberen.
	Het getal \pySnip{@float3.14@} is een \pySnip{float}, akkoord?
	Kunnen we dit omvormen naar een waarde van het type \pySnip{int} denk je?
	\newline
	Zullen we dat eens samen proberen?
\begin{pyEnv}
>>> int(@float3.14@)
\end{pyEnv}
\end{letsTryOut}

\subsection{Kort overzicht}

%%CHEAT_SHEET_START

\begin{itemize}
	\item
		\stringtype
		\newline
		Tekst, begint en eindigt met aanhalingstekens
		\newline
		vb. \pySnip{"Dit is een tekst"}
	\item
		\integertype
		\newline
		Gehele getallen (getallen zonder komma)
		\newline
		vb. \pySnip{47}
	\item
		\floattype
		\newline
		Getallen met komma
		\newline
		vb. \pySnip{@float3.14@}
	\item
		\booleantype
		\newline
		Logische waarden
		\newline
		\true of \false
	\item
		Gegevens omvormen van het de ene soort in de andere:
		\newline
		\pySnip{str(_)}, \pySnip{int(_)}, \pySnip{float(_)}, \pySnip{bool(_)}
\end{itemize}