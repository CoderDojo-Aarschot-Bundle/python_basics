\section{Variabelen}
\label{sec:variabelen}

Wanneer we een programma maken, dan gaan we dit vaak in kleine stapjes opbouwen.
Heel vaak hebben we het resultaat van een vorige stap nodig in de volgende.
Dus moeten we dit resultaat (tijdelijk) kunnen onthouden tussen de verschillende stappen.
Dit doen we door gebruik te maken van variabelen.

\subsection{Wat is een variabele?}

Een variabele is een soort doosje waarin we gegevens kunnen opslaan.
We kunnen dit doosje een naam geven, zodat we het later nog weten wat er in zit.
Zo een doosje kan ook leeg zijn, in python wordt dit aangegeven door het woordje \pySnip{None}.
\par
De volgende figuur toont een voorbeeld van vier variabelen die een paar eigenschappen van een jongen/meisje bijhouden.
\begin{itemize}
	\item
		\ref{variabelsExample1.name}: een variable \pySnip{naam: str} voor je naam.
	\item
		\ref{variabelsExample1.age}: een variable \pySnip{leeftijd: int} voor je leeftijd.
	\item
		\ref{variabelsExample1.outdoorPlayer}: een variable \pySnip{isBuitenSpeler: bool} om aan te geven of je graag buiten speelt.
	\item
		\ref{variabelsExample1.company}: een variable \pySnip{werkgever: str | None} om bij te houden waar je werkt,
		de waarde None betekent dat dit doosje leeg is (dus dat je nog niet werkt).
\end{itemize}

\begin{figure}[H]
	\centering
	\begin{subfigure}[b]{.2\textwidth}
		\centering
		\scalebox{.7}{
			\begin{tikzpicture}
				\variabeleBox{naam}{"Maxime"}
			\end{tikzpicture}
		}
		\caption{}
		\label{variabelsExample1.name}
	\end{subfigure}
	\begin{subfigure}[b]{.2\textwidth}
		\centering
		\scalebox{.7}{
			\begin{tikzpicture}
				\variabeleBox{leeftijd}{11}
			\end{tikzpicture}
		}
		\caption{}
		\label{variabelsExample1.age}
	\end{subfigure}
	\begin{subfigure}[b]{.2\textwidth}
		\centering
		\scalebox{.7}{
			\begin{tikzpicture}
				\variabeleBox{isBuitenSpeler}{True}
			\end{tikzpicture}
		}
		\caption{}
		\label{variabelsExample1.outdoorPlayer}
	\end{subfigure}
	\begin{subfigure}[b]{.2\textwidth}
		\centering
		\scalebox{.7}{
			\begin{tikzpicture}
				\variabeleBox{werkgever}{None}
			\end{tikzpicture}
		}
		\caption{}
		\label{variabelsExample1.company}
	\end{subfigure}
	\caption{
		Een voorbeeld van vier variabelen die een paar eigenschappen van een jongen/meisje bevatten.
	}
\end{figure}

\subsection{Spelen met variabelen}

\subsubsection{Een nieuwe variabele maken}

In python gebruiken we een isgelijkteken \pySnip{=} om een variabele een nieuwe waarde te geven.
De naam van de waarde komt voor het isgelijkteken en kan bestaan uit
(hoofd)letters (A/a - Z/z), cijfers (0 - 9, maar niet het eerste karakter van de naam) en een liggend streepje (\_).
Als er geen variabele bestaat met de gegeven naam, dan wordt er een nieuwe gemaakt met deze naam,
en krijgt deze de waarde die je rechts van het isgelijkteken staan hebt.

\subsubsection{Een nieuwe waarde voor een bestaande variabele}

Eenmaal een doosje bestaat,
dan ga zijn inhoud regelmatig veranderen door een nieuwe waarde.
Gelukkig kunnen we heel gemakkelijk zo een nieuwe waarde toekennen aan een bestaande variabele!
Eigenlijk is dat juist hetzelfde als wanneer we de variabele de eerste keer gemaakt hebben,
ook gewoon met het \pySnip{=} teken.
Als de variabele deze keer al bestaat,
dan wordt de waarde in dit doosje gewoon vervangen en wordt er geen nieuw doosje aangemaakt.

\subsubsection{De waarde uit een variabele gebruiken}

Wanneer we een doosje een inhoud geven,
dan doen we dat natuurlijk omdat we daar later iets mee willen doen.
Waardes kunnen we al gebruiken,
en dat doen we gewoon door ze te typen waar we ze nodig hebben, toch?
Wel, voor variabelen doen we gewoon hetzelfde:
we typen gewoon de variabele waar we ze nodig hebben.
Simpel he!

\begin{letsTryOut}
	Hier vind je twee voorbeelden die beide 'hallo'| lijken te zullen tonen.
	Denk je dat beide voorbeelden hetzelfde zullen doen, waarom wel/niet?
	\newline
	Kan jij deze voorbeelden eens uitproberen?
\begin{pyEnv}
>>> print("hallo")
\end{pyEnv}
\begin{pyEnv}
>>> hallo = "bonjour"
>>> print(hallo)
\end{pyEnv}
	Oh ja, \verb|'bonjour'| is het Franse woord voor \verb|'goeiedag'|.
\end{letsTryOut}

\subsubsection{Een groter voorbeeldje}

Zullen we nu eens verschillende variabelen maken, aanpassen en gebruiken?
En dat allemaal in \'e\'en groot voorbeeld?!
Neem zeker eens een kijkje naar het stukje code hieronder.

\begin{pyEnv}
naam = "Maxime"             # een nieuwe variabele 'naam'
                            # met de waarde 'Maxime'

leeftijd = 11               # een tweede nieuwe variabele 'leeftijd'
						    # met de waarde '11'

activiteit = "kamp bouwen"  # een derde nieuwe variabele 'activiteit'
							# voor de huidige bezigheid van Maxime

activiteit = None           # Maxime verveelt zich,
                            # en heeft dus niets te doen
							# -> het doosje moet nu dus leeg zijn
							#    dat is waarvoor we 'None'
							#    gebruiken in python

activiteit = "avond eten"   # etenstijd

leeftijd = leeftijd + 1     # Maxime is jarig!
                            # de nieuwe waarde voor het doosje
							# 'leeftijd' is de huidige waarde plus 1

print(f"{naam} is ondertussen al {leeftijd} jaar!")
\end{pyEnv}

\begin{letsTryOut}
	Welke leeftijd zou er getoond worden door de \pySnip{print()} onderaan uit bovenstaand stukje code?
	Is dat 11 jaar, de waarde die we er in het begin in de variabele hebben ingestoken?
	\newline
	Tip: je kan de code uit het voorbeeld hierboven kopiëren en plakken in de interactive shell.
\end{letsTryOut}

\begin{firstAidToErrors}{NameError}
	Werkt je code niet en je krijgt in de plaats output in de vorm van:
	\newline
	\verb|NameError: name 'abc' is not defined|
	\newline
	Dat de manier waarop python je vertelt dat hij de variabele \verb|abc| niet kent.
	Misschien heb je een typfoutje gemaakt in de naam van je variabele?
	Controleer of je de naam van je variabele overal hetzelfde hebt geschreven.
	En probeer daarna of het wel werkt.
\end{firstAidToErrors}

\subsection{Kort samengevat}

%%CHEAT_SHEET_START

\begin{itemize}
	\item
		Een variabele is een doosje waarin we gegevens kunnen opslaan.
		\begin{figure}[H]
			\centering
			\scalebox{.7}{
				\begin{tikzpicture}
					\variabeleBox{naam}{"Maxime"}
				\end{tikzpicture}
			}
		\end{figure}
	\item
		\pySnip{mijnVariabele = <waarde>}
		\newline
		Een nieuwe variabele maken we met het isgelijkteken \pySnip{=}.
		De naam van de variabele komt voor het isgelijkteken en de waarde die we er in willen stoppen komt er achter.
	\item
		\pySnip{mijnVariabele = <nieuweWaarde>}
		\newline
		We kunnen een bestaande variabele ook een nieuwe waarde geven.
		Dit doen we net op dezelfde manier zoals we een nieuwe variabele maken.
		Maar nu bestaat de variabele al en wordt de waarde in het doosje gewoon vervangen.
	\item
		\pySnip{print(mijnVariabele)}
		\newline
		De waarde in een variabele kan je gebruiken door de naam van de variabele te typen
		op de plaats waar je deze waarde nodig hebt.
		Deze waarde kan je dan gebruiken zoals je dat met een gewone waarde zou doen.
	\item
		\pySnip{mijnVariabele = None}
		\newline
		Moet het doosje leeg zijn?
		Dan gebruik je de waarde \pySnip{None}.
		In python geeft de speciale waarde \pySnip{None} aan dat het doosje leeg is.
\end{itemize}