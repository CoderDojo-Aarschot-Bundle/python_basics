\section{Herbruikbare blokken}

% In de meest eenvoudige programma's worden de regels code gewoon een na een uitgevoerd in de volgorde dat ze in het programma staan.
% Voor grotere programma's is dat niet altijd de beste aanpak.
% Gelukkig hebben we als programmeur een hoop hulpmiddelen waarmee we de volgorde van de uitvoering kunnen manipuleren.
% Herinner je nog het \pySnip{if}-statement (uit hoofdstuk \ref{sec:voorwaardelijke-uitvoering})
% dat het mogelijk maakte om blokken code over te slaan?
% Dit is misschien wel het eenvoudigste hulpmiddel dat we hebben.
% Ja maar toch niet het enige, hoor ik je al luidop denken.
% Natuurlijk heb je gelijk, het \pySnip{while}-statement (hoofdstuk \ref{sec:voorwaardelijke-lus}),
% waarmee we eenzelfde blok code meermaals achtereen konden laten uitvoeren,
% is ook een goed voorbeeld.
% \par
% Een nieuw, heel krachtig hulpmiddel zijn functies.
% Een functie is blokje code dat niet meteen uitgevoerd wordt wanneer het programma hierover gaat,
% maar waar we later naar kunnen verwijzen om elders in het programma uitgevoerd te worden.
% Waarom zouden we nu ooit code schrijven op de ene plaats en het op een andere plaats uitvoeren,
% dat maakt het programma toch kei ingewikkeld om te verstaan?!
% Nee, helemaal niet, het maakt de code juist veel leesbaarder!
% Er zijn verschillende rederen waarom we functies willen gebruiken:
% \begin{itemize}
% 	\item
% 		We hebben een stukje code dat we meermaals willen hergebruiken in het programma.
% 		Maar deze keer op verschillende plaatsen, dus niet meteen achtereen zoals het geval is bij de \pySnip{while}-lus.
% 		Eigenlijk is het al een goed idee om meteen een functie te gebruiken vanaf het moment dat je een stukje code een tweede keer wil hergebruiken.
% 	\item
% 		We kunnen ook functies gebruiken voor code die we maar \'e\'en keer gaan gebruiken in het programma.
% 		Dat doen we meestal om de leesbaarheid van het programma te verbeteren.
% 		Dat kan omdat functies een naam kunnen geven,
% 		en een goed gekozen naam maakt duidelijker wat de bedoeling van het stukje code is.
% \end{itemize}

% \subsection{Eenvoudige functies}

% Een functie is een blok code dat we eerst defini\"eren en het daarna pas elders in het programma uit te voeren.
% Het \pySnip{def}-statement geeft aan dat we een functie gaan defini\"eren.
% We defini\"eren, en gebruiken, een functie als volgt:
% \begin{pyEnv}
% def duidelijkeFunctienaam():
% 	# code die pas uitgevoerd wordt
% 	# wanneer we later deze functie aanroepen

% # hieronder wordt de functie opgeroepen
% # en de code in de functie uitgevoerd
% duidelijkeFunctienaam()
% \end{pyEnv}
% Laten we daar meteen een voorbeeld van bekijken.
% We gaan een functie defini\"eren die de huidige tijd toont.
% Deze functie gaan we daarna een paar keer oproepen,
% telkens met een seconde ertussen.
% \begin{pyEnv}
% # we gaan bestaande code gebruiken om de huidige tijd vinden
% from time import time, sleep

% def toonHuidigeTijd():
% 	nu = time()             # we maken een variabele
% 	                        # en slaan de huidige tijd erin op
% 							# dit zijn al de seconden
% 							# sinds 1 januari 1970
% 	nu = int(nu)            # en we ronden meteen af op een geheel getal
% 	                        # wat na de komma is, hebben we niet nodig

% 	minuten1970 = nu//60    # het aantal minuten sinds 1 januari 1970
% 	                        # // is gedeeld door, maar dan zonder rest
	
% 	minutenDag = minuten1970 % 1440  # nu hebben we enkel de minuten
% 	                        # sinds het begin van de huidige dag
% 	                        # % geeft de rest van de deling
% 	                        # 1440: aantal minuten in een dag (24*60)

% 	uren, minuten = divmod(minutenDag, 60)
% 	                        # divmod is de deling met quotient en rest
% 							# dus eigenlijk '//' en '%' gecombineerd
% 	                        # uren is het quotient, minuten is de rest

% 	# tot slot tonen we de tijd
% 	# met eventueel een nul vooraan zodat er altijd twee cijfers zijn
% 	print(f"\r{uren:02d}u{minuten:02d}", end = "")

% toonHuidigeTijd()           # we roepen de functie aan
% sleep(1)                    # we wachten een seconde
% toonHuidigeTijd()           # we roepen de functie opnieuw aan
% sleep(1)                    # we wachten opnieuw een seconde
% toonHuidigeTijd()           # we roepen de functie nog eens aan
% \end{pyEnv}
% Dat zijn meteen heel wat regels code om de huidige tijd te tonen.
% Door deze regels te combineren in \'e\'en functie,
% hoeven we dit stukje code maar \'e\'en keer te implementeren,
% en kunnen we dat daarna zoveel keer, maar echt zo veel keer (!), gebruiken als we maar willen.

% \begin{brainy}
% 	Heb je ook iets vreemd opgemerkt bij de \pySnip{print} in het voorbeeld hierboven?
% 	We hebben daar
% 	\begin{itemize}
% 		\item
% 			het \pySnip{end = '...'}-argument gebruikt.
% 			\newline
% 			Hierdoor hebben we het normale gedrag van \pySnip{print} aangepast.
% 			Ipv. een nieuwe regel te starten na de \pySnip{print},
% 			wordt er nu niets gedaan na de \pySnip{print}.
% 			Nu wordt de volgende \pySnip{print}meteen op dezelfde regel getoond.
% 		\item
% 			het \pySnip{'\\r'}-karakter gebruikt.
% 			\newline
% 			Dit zorgt ervoor dat de cursor terug helemaal naar het begin van de regel gaat.
% 	\end{itemize}
% 	Samen zorgen deze twee ervoor dat we steeds de vorige tijd overschrijven.
% 	Zo krijgen we een klok-effect.
% \end{brainy}

\begin{pyEnv}
# start van het spel
score = 0

# toon de score (als twee karakters lang)
print("Huidige score: ", end = "")
if score < 10:
	print("0", end = "")
print(str(score))

# je hebt geluk: je krijgt meteen wat punten cadeau!
score += 7

# toon de nieuwe score (opnieuw als twee karakters lang)
print("Huidige score: ", end = "")
if score < 0:
	print("0", end = "")
print(str(score))

# bonus verdient, goed bezig!
score += 5

# nieuwe score, dus weeral tonen
print("Huidige score: ", end = "")
if score < 0:
	print("0", end = "")
print(str(score))

# enz. ...
\end{pyEnv}
Interessant spel he?
We zijn al meteen goed bezig!
Maar er lijken wel veel regels code te zijn die telkens terugkomen.
Dat kunnen we vast efficiënter doen.
Jazeker, welkom in de wondere wereld van de functies!% (incl. bonuslevels)!
\par
Een functie is blokje code dat we schrijven,
maar dat niet meteen uitgevoerd mag worden.
Als we later in het programma naar deze functie verwijzen,
dan wordt de code in de functie pas uitgevoerd.
Nu hoor ik je al denken: "Maken we het zo niet gewoon ingewikkeld?".
Nee, helemaal niet, het maakt de code juist veel leesbaarder!
Functies zullen ons o.a. helpen als we:
\begin{itemize}
	\item
		een stukje code niet \'e\'e keer,
		maar meermaals willen hergebruiken in het programma.
	\item
		de leesbaarheid van het programma te verhogen.
		Dat kunnen we omdat we functies een naam geven.
		Met deze naam, kan je samenvatten wat een hele blok code eigenlijk doet.
	\item
		Je gaat vast nog voorbeelden vinden waarbij je een functie zal willen gebruiken!
\end{itemize}

\subsection{Eenvoudige functies}

Een functie is een blok code dat we eerst defini\"eren en het daarna pas elders in het programma uit te voeren.
Het \pySnip{def}-statement geeft aan dat we een functie gaan defini\"eren.
We defini\"eren, en gebruiken, een functie als volgt:
\begin{pyEnv}
def duidelijkeFunctienaam():
	# code die pas uitgevoerd wordt
	# wanneer we later deze functie aanroepen

# hieronder wordt de functie opgeroepen
# en de code in de functie uitgevoerd
duidelijkeFunctienaam()
\end{pyEnv}
We hadden gezegd dat we het voorbeeldje hierboven zouden kunnen inkorten
wanneer we functies kunnen toepassen, niet?
Jawel, dan gaat we dat ook meteen eens proberen:
\begin{pyEnv}
# we maken de functie om de score te tonen
# deze code in de functie wordt voorlopig overgeslagen
def toonScore():
	# toon de score (als twee karakters lang)
	print("Huidige score: ", end = "")
	if score < 10:
		print("0", end = "")
	print(str(score))

score = 0                   # start van het spel

toonScore()                 # toon de initiele score

score += 7                  # je hebt geluk: meteen wat punten cadeau!

toonScore()                 # toon de nieuwe score

score += 5                  # bonus verdient, goed bezig!

toonScore()                 # nieuwe score, dus weeral tonen

# enz. ...
\end{pyEnv}
Dat is nu toch veel leesbaarder, he.
\newline
Mission completed: we kunnen functies maken en gebruiken!

\begin{letsTryOut}
	Oei, aan het tempo dat we bezig zijn,
	gaan rap scores boven de 100 hebben!
	\newline
	We zouden de score als 3 karakters ipv. slechts twee moeten tonen.
	Ik heb iets gevonden dat lijkt te werken:
\begin{pyEnv}
if score < 100:
	print("0", end = "")
toonScore()
\end{pyEnv}
	Als ik dit \pySnip{if}-statement overal kopi\"eer voor iedere \pySnip{toonScore()},
	dan lijkt ons probleem opgelost, oef.
	\newline
	Klopt dat denk je, zal de score nu correct als 3 karakters worden getoond?
	\newline
	Probeer het zeker eens uit!
	\newline
	\newline
	Maar is dat nu wel de juiste oplossing?
	Ik blijf maar denken dat het beter kan,
	maar kan zelf niet vinden hoe.
	\newline
	Kan jij een betere oplossing voorstellen?
	\newline
	Laten we jouw oplossing ook eens uitproberen!
\begin{pyEnv}
>>> ... # jouw oplossing, ik ken ze nog niet he ;)
\end{pyEnv}
\end{letsTryOut}

\begin{brainy}
	\pySnip{print(_, end = '...')}
	\newline
	\newline
	Had je deze \pySnip{end = '...'} ook gezien in het voorbeeld?
	\newline
	Hierdoor hebben we het normale gedrag van \pySnip{print} aangepast.
	Ipv. een nieuwe regel te starten na de \pySnip{print},
	wordt er nu niets gedaan na de \pySnip{print}.
	De volgende \pySnip{print} wordt dus meteen op dezelfde regel getoond.
\end{brainy}

\subsection{Functies met argumenten}

In een functie defini\"eren we code dat we meermaals willen hergebruiken.
Herinner je dat we soms variabelen maken om later in de code te kunnen gebruiken
(hoofdstuk \ref{sec:variabelen})?
Zo gaan we ook code willen hergebruiken dat iets doet met een waarde.
We gaan deze code willen kunnen oproepen, maar telkens met een andere waarde(s) als uitgangspunt.
En ook dat kunnen we met functies.
Al moeten we deze waarde(s) nu wel extra kunnen meegeven: dat gaat via functie-argumenten.
We doen dit als volgt:
\begin{pyEnv}
def functienaam(arg0@optional: type@, arg1@optional: type@, ...):
	# code van de functie
	# * binnen de functie kunnen we de argumenten
	#   gebruiken als gewone variabelen
	# * ': type' mag je gebruiken om aan te geven
	#   wat voor soort gegevens de argumenten zijn
	#   maar het is niet verplicht

# hieronder roepen we de functie aan
# en geven we waardes mee aan de argumenten
functienaam(waarde0, waarde1, ...)
\end{pyEnv}
Ben je al helemaal mee?
Ik nog niet helemaal hoor.
Maar da's niks, een voorbeeld doet wonderen.
We gaan nog eens de score tonen.
Deze keer gaan we het maximum er ook bij tonen.
Het tonen van de score en het maximum gaan we telkens op dezelfde manier tonen.
Dat zal dus een functie worden.
Maar de score, alsook het maximum, zijn waardes die kunnen veranderen.
Deze zullen we dus moeten meegeven als argumenten.
We zouden bijvoorbeeld de functie zo kunnen opbouwen:
\begin{pyEnv}
def toonScore(score@optional: int@, maximum@optional: int@):
	print(f"Huidige score: {score:3d}/{maximum:d}")

# voorbeeld van het aanroepen van de functie
toonScore(47, 50)  # de speler heeft bijna het maximum!
\end{pyEnv}
Hierin worden de score en het maximum als argumenten meegegeven.
Daarna worden ze als zijnde variabelen gebruikt binnenin de functie.
Als je deze zin leest, dan heb je alles gelezen van functies met argumenten.
Je met glans en glorie geslaagd!
Op naar de volgende ronde.

\subsection{Het \pySnip{return}-statement}

6u58 \dots 6u59 \dots 7u00 BIEB BIEB BIEB!
\newline
Hey, het is wel zaterdag he!
\newline
Geen school vandaag, vlug snoozen en blijf ik nog vijf minuten liggen.
\newline
\par
Herken je deze situatie?
Natuurlijk niet, het is weekend, een ganse dag om te ravotten!
Dan heb ik toch geen wekker nodig om op te staan, duh.
Maar toch, kan je geloven dat sommigen toch eerst nog eens snoozen alvorens op te staan?
Hoe zou de wekker dat afhandelen?
\par
Snoozen wil zeggen: de wekker stopt en gaat vijf minuten later opnieuw af.
De nieuwe tijd is dus vijf minuten na het moment dat onze schone slaper op snooze duwt.
De nieuwe tijd berekenen is een taak dat een wekker wel vaker moet doen,
een functie dus!
\newline
Alhoewel, het resultaat van de berekening hebben we wel nodig.
Wanneer een functie eindigt,
dan spring de uitvoering van het programma terug naar de plaats waar de functie opgeroepen was.
Zijn we dan het resultaa gewoon kwijt?
Helemaal niet, we kunnen het resultaat van een functie teruggegeven aan het programma.
Dat is waarvoor we het \pySnip{return}-statement gebruiken:
\begin{pyEnv}
def functieMetResultaat(arg0@optional: type@, arg1@optional: type@, ...)@optional -> type@:
    # de code van de functie
	# ditmaal met een waarde
	# die teruggegeven moet worden
	return waarde           # een constante waarde
	                        # of een variabele
	# de functie wordt onheroepelijk gestopt
	# na het return-statement
	# code erna wordt NIET meer uitgevoerd

# nu kunnen we de functie oproepen
# en gebruiken op elke plaats
# waar we ook een waarde kunnen gebruiken
# vb: hier bewaren we het resultaat in een variabele
variabele = functieMetResultaat(waarde0, waarde1, ...)
\end{pyEnv}
En zo gaan we het \pySnip{return}-statement meestal gebruiken.
Maar daarnaast zullen deze weetjes je ook zeker helpen:
\begin{itemize}
	\item
		De functie stopt meteen na het \pySnip{return}-statement.
		Iedere regel code na een \pySnip{return}-statement wordt niet (maar echt niet he) uitgevoerd.
		Het verloop van het programma springt meteen terug naar de plaats waar de functie werd opgeroepen.
	\item
		Het \pySnip{return}-statement kan alleen maar gebruikt worden binnen een functie.
		Als je \pySnip{return} gebruikt buiten een functie,
		dan zal het programma eindigen met een error daar waar het \pySnip{return}-statement
		foutief gebruikt wordt.
	\item
		Ja hoeft niet altijd een waarde mee te geven met een \pySnip{return}-statement.
		Je kan het \pySnip{return}-statement ook gebruiken zonder een waarde mee te geven.
		Dan zorgt het \pySnip{return}-statement er gewoon voor dat de functie stopt.
		Dit kan handog zijn in een \pySnip{if}, \pySnip{else}, \pySnip{while} blok.
\end{itemize}
Allemaal goed en wel, maar hoe doen we dat nu?
Wel, de functie die onze wekker gebruikt,
zou er weleens zo uit kunnen zien:
\begin{pyEnv}
def tijdNaSnoozen(uren@optional: int@, minuten@optional: int@)@optional -> Tuple[int, int]@:
    minuten += 5            # we moeten er 5 minuten bijtellen
	if minuten >= 60:
	    # we moeten de uren verhogen als er meer dan 60 minuten zijn
	    minuten -= 60
		uren += 1
	return uren, minuten

# nu gaan we de functie gebruiken
# en kunnen we de nieuwe tijd voor de wekker
uren, minuten = tijdNaSnoozen(7, 1)
print(f"Wekker gaat nu af om {uren:02d}:{minuten:02d}")
\end{pyEnv}

\begin{letsTryOut}
\begin{pyEnv}
>>> tijdNaSnoozen(7, 1)
>>> tijdNaSnoozen(7, 56)
\end{pyEnv}
	Klopt het resultaat?
	\newline
	Is dat altijd zo denk je?
	Dan heb je dit nog niet geprobeerd:
\begin{pyEnv}
>>> tijdNaSnoozen(23, 58)
\end{pyEnv}
	Kan jij de functie aanpassen zodat dit ook werkt?
\end{letsTryOut}

\subsection{Recursieve functies}

Zeer besten lezer \dots
\newline
Sluit de ogen en ontspan.
\newline
(En doe ze ook maar weer open, anders wordt het moeielijk lezen \dots)
\newline
Alleen zo zal de magie tot jou komen!
\newline
\newline
Welja, wat we nu gaan ontdekken is nu toch wel echt een klein beetje speciaal hoor.
Dus ik vond zo een intro wel plezant!
Misschien dacht je nu nog dat je een functie alleen maar kan gebruiken, pas nadat ze gedifini\"eert is?
Wel, dan gaat het volgende plezant worden!

\subsubsection{$n$ faculteit}

Stel dat we nu eens het product willen weten van alle getallen van 1 tot en met een gegeven getal $n$.
Bijvoorbeeld, we geven als input het getal 5, dan moet het resultaat 120 (= $1*2*3*4*5$) zijn.
Dit product wordt met een moeilijk woord 'faculteit' genoemd.
Zo is '5 faculteit' gelijk aan 120, of '2 faculteit' gewoon 2 (want $1*2 = 2$).
Een hele toffe eigenschap is dat als je '5 faculteit' kent, dan je dan heel eenvoudig '6 faculteit' kunt vinden.
Ah ja, '6 faculteit' is gelijk aan 6*'5 faculteit' ($6*5*4*3*2*1 = 6*(5*4*3*2*1)$) oftewel 720.
En dat is een algemene eigenschap, '$n$ faculteit' is gelijk aan $n$*'$n - 1$ faculteit'
\par
We kunnen nu de functie \pySnip{def faculteit(n@optional: int@)@optional -> int@} maken.
Maar we gaan niet het hele product uit rekenen in de functie, we gaan lui zijn!
We gaan gewoon $n$ vermenigvuldigen met het resultaat van \pySnip{faculteit(n - 1)}.
En we weten dat die functie \pySnip{faculteit(_)} bestaat, want die zijn we nu aan het maken!
Een beetje verward, niet?
Geen zorgen, laten we het gewoon eens proberen:
\begin{pyEnv}
def faculteit(n@optional: int@)@optional -> int@:
	if n == 1:              # triviaal geval
		                    # we moeten kunnen stoppen op het eind
		return 1            # '1 faculteit' is 1
	else:                   # recursief geval
		return n*faculteit(n - 1)  # n faculteit = n*'(n - 1) faculteit'
\end{pyEnv}
Een functie die zichzelf aanroept, dat noemen we nu een recursieve functie.

\begin{letsTryOut}
	Nooit zomaar geloven wat je gezegt wordt, he!
	Niets zegt dat dit echt werkt.
	Er is maar \'e\'en manier om daar achter te komen!
	\newline
	We proberen het gewoon uit!
	Wat denk je dat het volgende gaat geven?
\begin{pyEnv}
>>> faculteit(1)
>>> faculteit(5)
>>> faculteit(6)
>>> faculteit(20)
\end{pyEnv}
	Zaten we juist?
\end{letsTryOut}

\begin{letsTryOut}
	Waarom hebben we eigenlijk \pySnip{if n == 1: return 1} gebruikt?
	Zo het niet werken zonder dat stukje code?
	Denk je dat deze functie ook gaat werken?
\begin{pyEnv}
def faculteit(n@optional: int@)@optional -> int@:
	return n*faculteit(n - 1)
\end{pyEnv}
	Probeer het eens uit!
	\newline
	Wat zie je gebeuren?
	En waarom denk je dat dit gebeurt?
\end{letsTryOut}

\subsubsection{De reeks van Fibonacci}

Herinner je je de reeks van Fibonacci nog uit hoofdstuk \ref{sec:voorwaardelijke-lus}?
Of heb je dat hoofdstuk stiekem overgeslagen?
In dat geval zullen hem eventjes opfrissen.
\par
De reeks van Fibonacci is een getallenrij waarbij ieder getal de som is van de twee voorgaande getallen.
De rij begint met tweemaal het getal 1.
Als we dat toepassen, dan krijgen we voor de eerste getallen: 1, 1, 2, 3, 5, 8, 13, 21, 34, ... .
\newline
Oftewel, het $n^{de}$ Fibonacci getal is de som van het $(n - 1)^{ste}$ Fibonacci getal en het $(n - 2)^{de}$ Fibonacci getal.
En met de bijkomende criteria dat zowel het eerste als het tweede Fibonacci getallen beide 1 zijn.
\par
Wel, voor de reeks van Fibonacci kunnen we ook handig gebruik maken van recursie.
Maar nu we gaan we het voor de verandering eens omgekeerd doen:
kan jij me het voorbeeld maken en er zo voor zorgen dat ik het ook snap?
\begin{letsTryOut}
	We gaan proberen (en slagen!) de volgende functie maken:
\begin{pyEnv}
def fibonacci(n@optional: int@)@optional -> int@:
	...
\end{pyEnv}
	Wat hebben we nodig?
	\begin{itemize}
		\item
			Het eerste element is $1$.
			\newline
			$\Rightarrow$ \pySnip{fibonacci(1) == 1}
		\item
			Het tweede element is ook $1$.
			\newline
			$\Rightarrow$ \pySnip{fibonacci(2) == 1}
		\item
			Elk volgend element is de som van de twee voorgaande elementen.
			\newline
			$\Rightarrow$ \pySnip{fibonacci(n) == fibonacci(n - 2) + fibonacci(n - 1)}
	\end{itemize}
	Kan jij de functie afmaken?
	\newline
	Is het je gelukt denk je?
\begin{pyEnv}
>>> fibonacci(1)
>>> fibonacci(5)
>>> faculteit(10)
\end{pyEnv}
	Krijg je het verwachte resultaat?
	\newline
	\newline
	Je mag zeker je oplossing tonen!
	Ook als je het niet helemaal kan vinden,
	het lukt ons vast en zeker samen wel!
\end{letsTryOut}

\subsection{Functies die je al gebruikt hebt}

Heb je dit hoofdstukje goed \textit{gesnopen}?
Dan is je misschien al opgevallen dat je (bestaande) functies eigenlijk al gebruikt hebt.
\begin{itemize}
	\item
		Zo is \pySnip{print(...)} een functie die we al heel vaak gebruikt hebben.
		Het is een functie die de argumenten die we meegeven toont op het scherm.
	\item
		Ook \pySnip{waarde = input(...)} is een functie die we al gebruikt hebben.
		Deze functie toont de tekst die we als argument meegeven op het scherm,
		en wacht daarna tot de gebruiker iets intypt en op enter drukt.
		De functie geeft dan de tekst die de gebruiker heeft ingetikt terug.
	\item
		Nog een ander voorbeeld zijn de functies \pySnip{int(...)}, \pySnip{float(...)}, \pySnip{str(...)}, \dots .
		Elk van deze functies probeert om de waarde, die we als argument meegeven, om te zetten naar het respectievelijke gegevens type.
	\item
		Tot slot zijn er nog zo vele andere functies die anderen voor ons al hebben gemaakt.
		Bijvoorbeeld, in het voorbeeld van de klok hebben we de functie \pySnip{time()} gebruikt.
		Deze functie komt uit de \pySnip{time}-module waar we eerst naar moesten verwijzen,
		anders had python niet geweten waar deze functie te vinden was.
\end{itemize}

\subsection{Een korte samenvatting}

%%CHEAT_SHEET_START

\begin{itemize}
	\item
		\pySnip{def functienaam(): ...}
		\newline
		Met het \pySnip{def}-statement defini\"er je een functie.
		De code in de functie wordt niet meteen uitgevoerd,
		maar pas wanneer je deze functie aanroept.
	\item
		\pySnip{functienaam()}
		\newline
		Door de functienaam te combineren met haakjes,
		roep je de functie aan.
		Nu wordt de code in de functie uitgevoerd.
	\item
		\pySnip{def functienaam(arg0@optional: type@, arg1@optional: type@, ...): ...}
		\newline
		Functies kunnen \'e\'en of meerdere argumenten hebben.
		Argumenten zijn waardes die je aan de functie doorgeeft.
		Binnen de functie kan je deze argumenten gebruiken als gewone variabelen.
	\item
		\pySnip{def functienaam()@optional -> returnType@: return ...}
		\newline
		Het \pySnip{return}-statement
		\begin{itemize}
			\item
				be\"eindigt de uitvoering van een functie meteen,
				zelfs als er nog code had achter gestaan.
			\item
				kan gevolgd worden door een waarde.
				Dan wordt deze waarde door de functie teruggegeven,
				en kan je verder gebruiken in de code. %op de plaats waar je de functie hebt aangeroepen.
			\item
				mag je alleen gebruiken binnen in een functie.
				% Als je \pySnip{return} gebruikt buiten een functie,
				% dan zal je een error krijgen.
		\end{itemize}
		% be\"eindigt de uitvoering van een functie meteen,
		% zelfs als er nog regels code hadden achter gestaan.
		% \newline
		% Als er een waarde achter \pySnip{return} staat,
		% dan wordt deze waarde door de functie teruggegeven.
		% Je kan deze waarde verder gebruiken op de plaats waar je de functie hebt aangeroepen.
		% \newline
		% Je mag het \pySnip{return}-statement alleen maar gebruiken binnen in een functie.
		% Als je \pySnip{return} gebruikt buiten een functie,
		% dan zal je een error krijgen.
	\item
		\pySnip{def functienaam(arg00@optional: type@, arg1@optional: type@, ...)@optional -> returnType@: return ...}
		\newline
		Natuurlijk kan een functie zowel argumenten als een \pySnip{return}-waarde hebben.
	\item
		Enkele voorbeelden van functies die je waarschijnlijk al gebruikt hebt:
		\begin{itemize}
			\item
				\pySnip{print(...)}
				\newline
				toont die de argumenten die we meegeven op het scherm.
			\item
				\pySnip{waarde = input(...)}
				\newline
				toont de tekst van het argument op het scherm,
				en geeft de input van de gebruiker daarna terug als \pySnip{str}.
			\item
				\pySnip{int(...)}, \pySnip{float(...)}, \pySnip{str(...)}, \dots
				\newline
				probeert om de waarde die we als argument meegeven om te zetten naar het respectievelijke gegevens type.
		\end{itemize}
\end{itemize}